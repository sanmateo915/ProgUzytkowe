\documentclass[a4paper,12pt]{article}
\usepackage[MeX]{polski}
\usepackage[utf8]{inputenc}

%opening
\title{}
\author{}

\begin{document}

\maketitle

\begin{abstract}

\end{abstract}

\section{Tryb matematyczny}

Ułamek w tekście $ \frac{1}{x} $ 

Oto równanie $c^{2}=a^{2}+b^{2}$

Ułamek $$ \frac{1}{x} $$

Oto równanie $$c^{2}=a^{2}+b^{2}$$


Ułamek 

\begin{equation}
\frac{1}{x}
\label{eq:rownanie1}
\end{equation}

Oto równanie

\begin{equation}
$$c^{2}=a^{2}+b^{2}$$
\label{eq:rownanie2}
\end{equation}

\section{Indeks górny i dolny}

Indeks górny $$x^{y}  \  e^{x} \ A^{2 \times 2}$$

Indeks dolny$$ x_y \ a_{ij} $$

$$ \frac{2{k}}{2{k+2}} $$

$$ 2^\frac{x^{2}}{(x+2)(x-2)^{3}} $$

$$ \vec{x}=[x_1,x_2,...x_n]$$

\section{Duże operatory matematyczne}

$$\sum \ \sum_{i=1}^{10}x_{i} \ \prod \ \coprod \ \int \ \oint \ \bigcap \ \bigcup\ \bigscup \ \bigvee \ \bigwedge \ \bigodot \ \bigotimes \ \bigoplus \ \biguplus$$

\end{document}